\documentclass[a4paper,11pt]{article}
\usepackage[utf8]{inputenc}
\usepackage[T1]{fontenc}
\usepackage{lmodern}
\usepackage{amsmath,amssymb,amsfonts}
\usepackage{geometry}
\usepackage{array}
\usepackage{booktabs}

\geometry{margin=1in}

\begin{document}

\title{UART Communication Protocol for our Autonomous Vehicle}
\author{Team 10 - PREN2}
\date{\today}
\maketitle

\section{Introduction}
This document describes a 64-bit fixed-size UART protocol designed for our autonomous vehicle. Key requirements are:
\begin{itemize}
  \item Low overhead,
  \item Robustness with CRC error detection,
  \item Fixed frame size (2 read cycles for each frame),
  \item Ability to carry up to 50 bits of parameter/reserved data.
\end{itemize}

\bigskip
Each 64-bit frame is laid out as follows:
\[
\underbrace{\text{Addr}}_{2 \text{ bits}}
\;\|\;
\underbrace{\text{Command}}_{4 \text{ bits}}
\;\|\;
\underbrace{\text{Parameter/Reserved}}_{50 \text{ bits}}
\;\|\;
\underbrace{\text{CRC-8}}_{8 \text{ bits}}
\]

\section{Frame Definition}
A single UART frame is 64 bits. The fields are:

\begin{itemize}
  \item \textbf{Address (2 bits)}:
    \[
      \text{Addr} \in \{0b00,\, 0b01,\, 0b11\}
    \]
    Example address mapping:
    \begin{align*}
      0b00 &\rightarrow \text{Raspberry Hat},\\
      0b01 &\rightarrow \text{Motion Controller},\\
      0b11 &\rightarrow \text{Grip Controller}.
    \end{align*}

  \item \textbf{Command (4 bits)}:
    \[
      \text{Command} \in \{ 0x0,\dots,0xF \}
    \]
    Commands currently in use are documented in Table~\ref{tab:commands}.

  \item \textbf{Parameter/Reserved (50 bits)}:
    \[
      \underbrace{p_{49} \, p_{48} \, \dots \, p_{1} \, p_{0}}_{50 \text{ bits}}
    \]
    This field can encode various data types, including those listed in Table~\ref{tab:types} (e.g.\ \texttt{int16}, \texttt{float16}, flags, etc.) depending on the command. If fewer than 50 bits are needed, the unused bits can be reserved or zeroed.

  \item \textbf{CRC-8 (8 bits)}:
    \[
      \underbrace{c_{7}\,c_{6}\,\dots\,c_{1}\,c_{0}}_{8 \text{ bits}}
    \]
    This is a standard 8-bit CRC computed over the top 56 bits.
\end{itemize}

\section{Command Reference}

\begin{table}[h!]
\centering
\begin{tabular}{>{\ttfamily}c l l}
\toprule
\textnormal{Code} & \textnormal{Command} & \textnormal{Parameter(s)} \\
\midrule
0x0 & Move & int16: distance in cm (or 0/unused for infinite) \\
0x1 & Reverse & int16: distance in cm (or 0/unused for infinite) \\
0x2 & Turn & int16: degrees (positive=right, negative=left) \\
0x3 & Rotate & int16: degrees (default 180 if 0/unused) \\
0x4 & Stop Normal & (open) \\
0x5 & Stop Emergency & (open) \\
0x6 & Info & (open) \\
0x7 & Ping & (open) \\
0x8 & Error Check & (open) \\
\bottomrule
\end{tabular}
\caption{Command Codes, Names, and Parameter Definitions}
\label{tab:commands}
\end{table}

\section{Type Reference}

\begin{table}[h!]
\centering
\begin{tabular}{l l}
\toprule
\textbf{Type} & \textbf{Range / Description} \\
\midrule
\texttt{int16} & -32768 to 32767 \\
\texttt{int8} & -128 to 127 \\
\texttt{uint16} & 0 to 65535 \\
\texttt{uint8} & 0 to 255 \\
\texttt{fp16 / float16} & half precision \\
\texttt{fp8 / float8} & mini precision \\
\bottomrule
\end{tabular}
\caption{Possible Data Types for Parameters}
\label{tab:types}
\end{table}

\newpage
\section{CRC-8 Computation}

\subsection{Polynomial Definition}
The polynomial we'll use for \texttt{CRC-8} is:
\[
  x^8 + x^2 + x^1 + x^0,
\]
often represented in hexadecimal as \texttt{0x07}.

\subsection{Why This Polynomial?}
\label{sec:why_poly}

The polynomial
\[
P(x) \;=\; x^8 + x^2 + x + 1
\]
is known to be \emph{primitive} over the finite field \(\mathrm{GF}(2)\). This implies that the linear feedback shift register (LFSR) based on \(P(x)\) can generate any nonzero 8-bit sequence in a maximal cycle (of length \(2^8 - 1 = 255\)). In terms of CRC properties, using a primitive polynomial of degree 8 provides:

\begin{itemize}
  \item \textbf{Guaranteed detection of any single-bit error}: A one-bit error corresponds to the polynomial \(x^k\) for some \(k\). Because \(P(x)\) is primitive, \(x^k\) is never divisible by \(P(x)\) for \(0 \le k < 255\).
  \item \textbf{Guaranteed detection of many double-bit errors}: If two bits are in error, the error polynomial is \(x^m + x^n\). Since \(P(x)\) is primitive, it cannot divide \(x^m + x^n\) unless \(m=n\). Thus, distinct double-bit errors (i.e.\ \(m \neq n\)) are almost always detected.\footnote{Strictly speaking, \emph{no} CRC polynomial can detect \emph{all} double-bit errors at arbitrary distances, but a primitive polynomial has a higher minimum distance for typical codeword lengths and is exceptionally good for short frames like ours.}
  \item \textbf{Detection of bursts up to 8 bits long}: In general, an \(n\)-bit CRC can reliably detect bursts of up to \(n\) bits in length. Here, that means any burst error of up to 8 consecutive bits will \emph{not} yield a zero remainder when dividing by \(P(x)\).
\end{itemize}

\paragraph{Short Frame Advantage:}
Because our frames are only 64 bits, the chance of an undetected multi-bit error is further reduced. For random error patterns, the probability of an undetected error is at most \(1\!/\!2^8 = 1/256\). Actual practical detection is typically even better than the pure random guess model, especially for correlated or bursty errors, due to how the polynomial aligns with short data lengths.

\newpage
\subsection{Computation Steps (Bitwise)}
\begin{enumerate}
  \item Initialize the CRC register:
    \[
      \text{crc} \gets 0x00.
    \]
  \item Process the first 7 bytes (56 bits) in the frame (i.e.\ \texttt{Addr}, \texttt{Command}, and \texttt{Parameter/Reserved}).
  \item For each byte \texttt{b} among those 7 bytes:
    \[
      \text{crc} \gets \text{crc} \oplus \texttt{b}.
    \]
    Then, for each of the 8 bits in that byte:
    \[
      \begin{cases}
        \text{if } (\text{crc} \,\&\, 0x80) \neq 0:
            & \text{crc} \gets (\text{crc} \ll 1) \oplus 0x07 \\
        \text{else}:
            & \text{crc} \gets (\text{crc} \ll 1)
      \end{cases}
    \]
    (Mask \texttt{crc} with \texttt{0xFF} after each shift to keep it at 8 bits.)

  \item At the end of this 7-byte process, \texttt{crc} is the final CRC-8. Store it into the last byte of the 64-bit frame.
\end{enumerate}

\section{Transmit \& Receive Flow}

\subsection{Transmit Flow}
\begin{enumerate}
  \item Construct the 56 bits of data, for example:
  \[
    \text{Frame\_Hi} = (\text{Addr} \ll 62)
                      \;\vert\; (\text{Command} \ll 58)
                      \;\vert\; (\text{Param}_{50} \ll 8).
  \]
  \item Compute CRC-8 across these 56 bits, producing one byte.
  \item Append this \texttt{crc\_byte} as the final 8 bits.
  \item Send the resulting 64-bit frame via UART.
\end{enumerate}

\subsection{Receive Flow}
\begin{enumerate}
  \item Read the full 64 bits from UART (fixed-size).
  \item Separate the top 56 bits (\texttt{Addr + Command + Param/Reserved}) and the last 8 bits (the received CRC).
  \item Compute a new CRC-8 over the 56 bits.
  \item Compare the computed CRC with the received CRC:
    \[
      \text{if } \text{computedCRC} \neq \text{receivedCRC}
      \quad \Longrightarrow \quad
      \text{reject packet (error).}
    \]
  \item If the CRCs match, accept the frame and parse the 56 bits according to the protocol’s definitions.
\end{enumerate}

\end{document}
